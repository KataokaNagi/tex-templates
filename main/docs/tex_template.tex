% @file           tex_template.tex
% @brief          Template code of LaTeX
% @author         Kataoka Nagi
% @date           2020-12-24 17:19:52
% $Version:       1.1
% @par            History
%                 separate usepackages to styles
% Copyright (c) 2021 Kataoka Nagi
% - This src is released under the MIT License, see LICENSE.

\documentclass[a4paper, papersize]{jsarticle}
% [titlepage] でタイトルのみのページになる

\usepackage{base_kataoka-nagi}
\usepackage{docs_kataoka-nagi}


%%%%%%%%%%%%%%%%%%%%%%%%%%%%%%%%%%%%%%%%%%%%%%%%%%%%%%%%%%%%%%%%%%%%%%%%%%%%%%%%

% 目次なし用タイトル
% \title{
% \vspace{-3cm} % タイトルのマージン
% 【授業名】 第【01など】回 レポート課題 \\
% 【サブタイトル】}
% \author{AL18036 片岡 凪}
% \date{提出日 2020年12月18日}


% 目次つき用タイトル
\title{
\vspace{-2cm} % タイトルのマージン
【授業名】 第【01など】回 レポート課題 \\
【サブタイトル】}
\author{AL18036 片岡 凪 \thanks{芝浦工業大学 工学部 情報工学科 3年}}
\date{提出締切 2021年0月日 \\
提出日  2021年0月日}

\begin{document}

\maketitle

%%%%%%%%%%%%%%%%%%%%%%%%%%%%%%%%%%%%%%%%%%%%%%%%%%

% 目次
\setcounter{tocdepth}{2}
\tableofcontents
\newpage

%%%%%%%%%%%%%%%%%%%%%%%%%%%%%%%%%%%%%%%%%%%%%%%%%%

\begin{abstract}
  【概要】
\end{abstract}

%%%%%%%%%%%%%%%%%%%%%%%%%%%%%%%%%%%%%%%%%%%%%%%%%%%%%%%%%%%%%%%%%%%%%%%%%%%%%%%%

\begin{multicols}{2}
  % \setcounter{section}{-1}

  %%%%%%%%%%%%%%%%%%%%%%%%%%%%%%%%%%%%%%%%%%%%%%%%%%%%%%%%%%%%%%%%%%%%%%%%%%%%%%%%

  \section{実行環境}
  \subsubsection{java}
  \begin{itemize}
    \item javac 11.0.7
    \item java 11.0.7 2020-04-14 LTS
    \item Java(TM) SE Runtime Environment 18.9 (build 11.0.7+8-LTS)
    \item Java HotSpot(TM) 64-Bit Server VM 18.9 (build 11.0.7+8-LTS, mixed mode)
  \end{itemize}

  %%%%%%%%%%%%%%%%%%%%%%%%%%%%%%%%%%%%%%%%%%%%%%%%%%

  \subsubsection{ターミナル}
  \begin{itemize}
    \item cmd.exe
          \begin{itemize}
            \item Microsoft Windows Version 10.0.18363.836上で実行
          \end{itemize}
  \end{itemize}

  %%%%%%%%%%%%%%%%%%%%%%%%%%%%%%%%%%%%%%%%%%%%%%%%%%%%%%%%%%%%%%%%%%%%%%%%%%%%%%%%

  \section{設問}

  %%%%%%%%%%%%%%%%%%%%%%%%%%%%%%%%%%%%%%%%%%%%%%%%%%

  \subsection{設問1}
  \begin{quotation}
    【本文】\footnote{【氏名】先生の講義,【講義名】の資料「【資料名】」【ページ数】ページ,$<$ \url{【URL】} $>$ (2020年01月17日アクセス) より抜粋}

    【本文】\footnote{芝浦工業大学の学生および教職員向けポータルサイトScombの課題提出ページ「【ページ名】」,$<$ \url{【URL】} $>$ (2020年01月17日アクセス) より抜粋}
  \end{quotation}

  %%%%%%%%%%%%%%%%%%%%%%%%%%%%%%%%%%%%%%%%%%%%%%%%%%%%%%%%%%%%%%%%%%%%%%%%%%%%%%%%

  \section{目的}

  %%%%%%%%%%%%%%%%%%%%%%%%%%%%%%%%%%%%%%%%%%%%%%%%%%

  \subsection{研究の背景}
  【本文】

  %%%%%%%%%%%%%%%%%%%%%%%%%%%%%%%%%%%%%%%%%%%%%%%%%%

  \subsection{研究の目的}
  【本文】

  %%%%%%%%%%%%%%%%%%%%%%%%%%%%%%%%%%%%%%%%%%%%%%%%%%%%%%%%%%%%%%%%%%%%%%%%%%%%%%%%

  \section{方法}

  %%%%%%%%%%%%%%%%%%%%%%%%%%%%%%%%%%%%%%%%%%%%%%%%%%

  \subsection{対象}
  【本文】

  %%%%%%%%%%%%%%%%%%%%%%%%%%%%%%%%%%%%%%%%%%%%%%%%%%

  \subsection{装置}
  【本文】

  %%%%%%%%%%%%%%%%%%%%%%%%%%%%%%%%%%%%%%%%%%%%%%%%%%

  \subsection{ソフトウェア}
  【本文】

  %%%%%%%%%%%%%%%%%%%%%%%%%%%%%%%%%%%%%%%%%%%%%%%%%%

  \subsection{ソースプログラム}

  %%%%%%%%%%%%%%%%%%%%%%%%%%%%%%

  \subsubsection{ソースコード}

  \setcounter{lstlisting}{4}
  \renewcommand{\lstlistingname}{ソースコード}
  \begin{lstlisting}[caption = 【ファイル名】, label=【参照名】]
【ソースコード】
\end{lstlisting}

  %%%%%%%%%%%%%%%%%%%%%%%%%%%%%%

  \setcounter{lstlisting}{2}
  \renewcommand{\lstlistingname}{実行結果}
  \subsubsection{実行結果}
  \setcounter{lstlisting}{0}
  \renewcommand{\lstlistingname}{実行結果}
  \begin{lstlisting}[caption = 【ファイル名】 \quad 【ファイル名】, label = 【参照名】]
\end{lstlisting}

  %%%%%%%%%%%%%%%%%%%%%%%%%%%%%%

  \begin{table}[H]
    \centering
    \caption{直感による順序付けとコサイン類似度による順序付けのまとめ}
    \label{order_table}
    \begin{tabular}{l||ccc}
      \hline
      名詞 & 直感 & コサイン類似度 & 順位の差 \\ \hline
      海賊 & 1位  & 1位            & 0        \\ \hline
      戦い & 2位  & 4位            & -2       \\ \hline
      秘宝 & 3位  & 3位            & 0        \\ \hline
      冒険 & 4位  & 2位            & 2        \\ \hline
      能力 & 5位  & 5位            & 0        \\ \hline
    \end{tabular}
  \end{table}

  %%%%%%%%%%%%%%%%%%%%%%%%%%%%%%%%%%%%%%%%%%%%%%%%%%

  \subsection{手続き}
  【本文】

  %%%%%%%%%%%%%%%%%%%%%%%%%%%%%%%%%%%%%%%%%%%%%%%%%%%%%%%%%%%%%%%%%%%%%%%%%%%%%%%%

  \section{仮説}
  【本文】

  %%%%%%%%%%%%%%%%%%%%%%%%%%%%%%%%%%%%%%%%%%%%%%%%%%%%%%%%%%%%%%%%%%%%%%%%%%%%%%%%

  \section{結果}
  【本文】

  %%%%%%%%%%%%%%%%%%%%%%%%%%%%%%%%%%%%%%%%%%%%%%%%%%%%%%%%%%%%%%%%%%%%%%%%%%%%%%%%

  \section{考察}

  %%%%%%%%%%%%%%%%%%%%%%%%%%%%%%%%%%%%%%%%%%%%%%%%%%

  \subsection{仮説の吟味}
  【本文】

  %%%%%%%%%%%%%%%%%%%%%%%%%%%%%%%%%%%%%%%%%%%%%%%%%%

  \subsection{その他の考察}
  【本文】

  %%%%%%%%%%%%%%%%%%%%%%%%%%%%%%%%%%%%%%%%%%%%%%%%%%%%%%%%%%%%%%%%%%%%%%%%%%%%%%%%

  \section{結論}
  【本文】
  ~\cite{【サイトの参照名】}.

  \renewcommand{\figurename}{図}
  \setcounter{figure}{2}
  \begin{figure}[H]
    \centering
    \includegraphics[width=65mm]{figures/Sample.jpg}
    \caption{【キャプション名】}
    \label{【ラベル名】}
  \end{figure}
  \noindent
  %%%%%%%%%%%%%%%%%%%%%%%%%%%%%%%%%%%%%%%%%%%%%%%%%%

  \begin{figure}[htpb]
    \centering
    \begin{tabular}{c}

      %-----left chart-----

      \begin{minipage}{0.47\hsize}
        \centering
        \includegraphics[keepaspectratio, height=40mm]
        {figures/Sample.png}
        \label{foo}
      \end{minipage}

      %--- 中央スペース

      \begin{minipage}{0.06\hsize}
        \hspace{2mm}
      \end{minipage}

      %-----right chart-----

      \begin{minipage}{0.47\hsize}
        \centering
        \includegraphics[keepaspectratio, height=40mm]
        {figures/Sample.png}
        \label{bar}
      \end{minipage} \\
    \end{tabular}
  \end{figure}

  %%%%%%%%%%%%%%%%%%%%%%%%%%%%%%%%%%%%%%%%%%%%%%%%%%%%%%%%%%%%%%%%%%%%%%%%%%%%%%%%

  \begin{thebibliography}{99}

    % F12 -> alert(document.lastModified); でサイトの更新日がわかる
    \bibitem{【サイトの参照名】}
    【人名】
    (2019)
    「【サイトのタイトル】」,
    \verb|<| \url{【URL】} \verb|>|
    2020年01月17日アクセス.

    \bibitem{【本の参照名】}
    【人名】,【人名】
    『【サイトのタイトル】』
    (【出版社】
    ,2017)

  \end{thebibliography}

  %%%%%%%%%%%%%%%%%%%%%%%%%%%%%%%%%%%%%%%%%%%%%%%%%%

\end{multicols}

\end{document}

\begin{comment}
メモ
\end{comment}
