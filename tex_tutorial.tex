%%%%%%%%%%%%%%%%%%%%%%%%%%%%%%%%%%%%%%%%%%%%%%%%%%%%%%%%%%%%%%%%%%%%%%%%%%%%%%%%
% 1 初期設定
%%%%%%%%%%%%%%%%%%%%%%%%%%%%%%%%%%%%%%%%%%%%%%%%%%%%%%%%%%%%%%%%%%%%%%%%%%%%%%%%

%%%%%%%%%%%%%%%%%%%%%%%%%%%%%%%%%%%%%%%%%%%%%%%%%%
% 1.1 文書の設定
%%%%%%%%%%%%%%%%%%%%%%%%%%%%%%%%%%%%%%%%%%%%%%%%%%

\documentclass[a4paper]{jsarticle}
% [titlepage] でタイトルのみのページになる

%%%%%%%%%%%%%%%%%%%%%%%%%%%%%%%%%%%%%%%%%%%%%%%%%%
% 1.2 LaTeXの追加機能の設定
%%%%%%%%%%%%%%%%%%%%%%%%%%%%%%%%%%%%%%%%%%%%%%%%%%

\usepackage[]{multicol} % n段組みの文章
\usepackage {amsmath} % 数式の利用
\usepackage{comment} % 複数行に渡るコメント
\usepackage{listings,jlisting} % 日本語のコメント
\usepackage{url} % URLをちゃんと出力
\usepackage[dvipdfmx, bookmarkstype=toc, colorlinks=false, pdfborder={0 0 0}, bookmarks=true, bookmarksnumbered=true]{hyperref} % ハイパーリンク
\usepackage{pxjahyper} % ハイパーリンク
\usepackage[dvipdfmx]{graphicx} % 画像の追加用
\usepackage{here} % 図を記述した場所に出力

%%%%%%%%%%%%%%%%%%%%%%%%%%%%%%%%%%%%%%%%%%%%%%%%%%
% 1.3 自作の設定
%%%%%%%%%%%%%%%%%%%%%%%%%%%%%%%%%%%%%%%%%%%%%%%%%%

% ここからソースコードの表示に関する設定
\lstset{
  basicstyle={\ttfamily},
  identifierstyle={\small},
  commentstyle={\smallitshape},
  keywordstyle={\small\bfseries},
  ndkeywordstyle={\small},
  stringstyle={\small\ttfamily},
  frame={tb},
  breaklines=true,
  columns=[l]{fullflexible},
  numbers=left,
  xrightmargin=0zw,
  xleftmargin=3zw,
  numberstyle={\scriptsize},
  stepnumber=1,
  numbersep=1zw,
  lineskip=-0.5ex
}
% ここまでソースコードの表示に関する設定
% \begin{lstlisting}[caption=hoge,label=fuga] を用いる
% キャプション名「ソースコードn」を「プログラムn」に変えるには,\begin{document}の前に\renewcommand{\lstlistingname}{プログラム}
% https://qiita.com/ta_b0_/items/2619d5927492edbb5b03

\def\bm#1{\mbox{\boldmath $#1$}} % for bm (vector)

%%%%%%%%%%%%%%%%%%%%%%%%%%%%%%%%%%%%%%%%%%%%%%%%%%
% 1.4 タイトルの設定
%%%%%%%%%%%%%%%%%%%%%%%%%%%%%%%%%%%%%%%%%%%%%%%%%%

% 目次なし用タイトル
% \title{
% \vspace{-3cm} % タイトルのマージン
% 【授業名】 第【01など】回 レポート課題 \\
% 【サブタイトル】}
% \author{AL18036 片岡 凪}
% \date{提出日 2020年12月18日}

% 目次つき用タイトル
\title{
\vspace{-1cm} % タイトルのマージン
\LaTeX mas\ Eve~ ~{\it 聖なる夜に君は美文書に浸る} \\
【アドカレ2020~24日目】}
\author{AL18036 片岡 凪 \thanks{芝浦工業大学 工学部 情報工学科 3年}}
\date{提出締切 2020年12月24日 \\
提出日  2020年12月24日}

%%%%%%%%%%%%%%%%%%%%%%%%%%%%%%%%%%%%%%%%%%%%%%%%%%%%%%%%%%%%%%%%%%%%%%%%%%%%%%%%
% 2. 本文
%%%%%%%%%%%%%%%%%%%%%%%%%%%%%%%%%%%%%%%%%%%%%%%%%%%%%%%%%%%%%%%%%%%%%%%%%%%%%%%%

%%%%%%%%%%%%%%%%%%%%%%%%%%%%%%%%%%%%%%%%%%%%%%%%%%
% 2.1 本文の開始地点を設定
%%%%%%%%%%%%%%%%%%%%%%%%%%%%%%%%%%%%%%%%%%%%%%%%%%

\begin{document}

%%%%%%%%%%%%%%%%%%%%%%%%%%%%%%%%%%%%%%%%%%%%%%%%%%
% 2.2 タイトルを出力
%%%%%%%%%%%%%%%%%%%%%%%%%%%%%%%%%%%%%%%%%%%%%%%%%%

\maketitle

%%%%%%%%%%%%%%%%%%%%%%%%%%%%%%%%%%%%%%%%%%%%%%%%%%
% 2.3 目次の出力
%%%%%%%%%%%%%%%%%%%%%%%%%%%%%%%%%%%%%%%%%%%%%%%%%%

\setcounter{tocdepth}{2}
\tableofcontents
\newpage

%%%%%%%%%%%%%%%%%%%%%%%%%%%%%%%%%%%%%%%%%%%%%%%%%%
% 2.4 概要
%%%%%%%%%%%%%%%%%%%%%%%%%%%%%%%%%%%%%%%%%%%%%%%%%%

\begin{abstract}
本稿では,デジクリ アドベントカレンダー2020 24日目\footnote{\url{https://core.digicre.net/blog/event/2/}} で紹介した,LaTeXの書き方の基本と応用について記述する.
\end{abstract}

%%%%%%%%%%%%%%%%%%%%%%%%%%%%%%%%%%%%%%%%%%%%%%%%%%
% 2.5 章ごとの中身を記述
%%%%%%%%%%%%%%%%%%%%%%%%%%%%%%%%%%%%%%%%%%%%%%%%%%

% 1章
\section{LaTeXの基本}

%%%%%%%%%%%%%%%%%%%%%%%%%%%%%%%%%%%%%%%%%%%%%%%%%%

% 1章1節
\subsection{機能の呼び出し方}
\begin{itemize}
\item \verb|\機能名|
\item \verb|\begin{機能名}[オプション]文章\end{機能名}|
\end{itemize}

%%%%%%%%%%%%%%%%%%%%%%%%%%%%%%%%%%%%%%%%%%%%%%%%%%

% 1章2節
\subsection{ソースファイルの大まかな構造}
\begin{itemize}
\item \verb|設定と本文と,それぞれの詳細|
\end{itemize}

%%%%%%%%%%%%%%%%%%%%%%%%%%%%%%%%%%%%%%%%%%%%%%%%%%

\subsection{改行の仕方}
\begin{itemize}
\item \verb|空行|
\item \verb|\\|
\item \verb|\noindent|
\end{itemize}

%%%%%%%%%%%%%%%%%%%%%%%%%%%%%%%%%%%%%%%%%%%%%%%%%%

\subsection{数式の出力}
\begin{itemize}
\item 文章中で$e^{i\theta} = cos{\theta} + i sin{\theta}$
\item 別行立てで
\end{itemize}
\begin{align}
  e^{i\theta}
  &= \cdots
  \notag
  &
  \\
  &= 
  cos{\theta} + i sin{\theta}
  &
  (答)
  \label{euler}
\end{align}

%%%%%%%%%%%%%%%%%%%%%%%%%%%%%%%%%%%%%%%%%%%%%%%%%%

\subsection{特殊文字の出力}
\begin{itemize}
\item 数式モードで使えるギリシャ文字など
\item \%や\&など
\item \verb|困ったら\verb|
\end{itemize}

%%%%%%%%%%%%%%%%%%%%%%%%%%%%%%%%%%%%%%%%%%%%%%%%%%
% 2.5.1 2コラム構成の開始地点を設定
%%%%%%%%%%%%%%%%%%%%%%%%%%%%%%%%%%%%%%%%%%%%%%%%%%

\newpage
\begin{multicols}{2}

%%%%%%%%%%%%%%%%%%%%%%%%%%%%%%%%%%%%%%%%%%%%%%%%%%

% 2章
\section{LaTeXの応用}

%%%%%%%%%%%%%%%%%%%%%%%%%%%%%%%%%%%%%%%%%%%%%%%%%%

\subsection{2段組み}
このページから始まったような2段組み

%%%%%%%%%%%%%%%%%%%%%%%%%%%%%%%%%%%%%%%%%%%%%%%%%%
\subsection{脚注}
前ページの下部にあるような脚注

\subsection{出力されないメモを書き残す}
\begin{itemize}
\item \verb|\%の利用|
\item \verb|\begin{comment}\end{comment}|
\end{itemize}

%%%%%%%%%%%%%%%%%%%%%%%%%%%%%%%%%%%%%%%%%%%%%%%%%%

\subsection{項目の列挙}
\begin{description}
\item [手法1.] アン
\item [手法2.] ドゥー
\item [手法3.] トロワ
\end{description}

%%%%%%%%%%%%%%%%%%%%%%%%%%%%%%%%%%%%%%%%%%%%%%%%%%

\subsection{図の挿入}
\renewcommand{\figurename}{図}
\setcounter{figure}{2}
\begin{figure}[H]
 \centering
   \includegraphics[width=20mm]{figures/Sample.jpg}
 \caption{わたし}
 \label{me}
\end{figure}
\noindent

%%%%%%%%%%%%%%%%%%%%%%%%%%%%%%%%%%%%%%%%%%%%%%%%%%

\subsection{表の挿入}

\begin{table}[H]
 \centering
 \caption{直感による順序付けとコサイン類似度による順序付けのまとめ}
 \label{order_table}
  \begin{tabular}{l||ccc}
  	\hline
   名詞 & 直感 & コサイン類似度 & 順位の差 \\ \hline
   海賊 & 1位 & 1位 & 0 \\ \hline
   戦い & 2位 & 4位 & -2 \\ \hline
   秘宝 & 3位 & 3位 & 0 \\ \hline
   冒険 & 4位 & 2位 & 2 \\ \hline
   能力 & 5位 & 5位 & 0 \\ \hline
  \end{tabular}
\end{table}

%%%%%%%%%%%%%%%%%%%%%%%%%%%%%%%%%%%%%%%%%%%%%%%%%%

\subsection{ソースコードの美しい挿入}

\setcounter{lstlisting}{1}
\renewcommand{\lstlistingname}{ソースコード}
\begin{lstlisting}[caption = hello\_world.c, label = hello]
int main(void) {
 printf("Hello World!");
}
\end{lstlisting}

%%%%%%%%%%%%%%%%%%%%%%%%%%%%%%%%%%%%%%%%%%%%%%%%%%

\subsection{目次やURLにハイパーリンクを張る}
\begin{itemize}
\item \verb|目次をクリックするとその場所に飛ぶ|
\item \verb|URLをクリックするとそのサイトが開く|
\item \verb|URLを記述する際の注意|
\end{itemize}


%%%%%%%%%%%%%%%%%%%%%%%%%%%%%%%%%%%%%%%%%%%%%%%%%%

\subsection{式や図表,参考文献の引用を示す}
式~(\ref{euler}),図~\ref{me},表~\ref{order_table},ソースコード~\ref{hello},参考文献~\cite{github}\cite{【本の参照名】}

%%%%%%%%%%%%%%%%%%%%%%%%%%%%%%%%%%%%%%%%%%%%%%%%%%

\subsection{加減乗除}
$+,~ -,~ \cdot,~ \times,~ \div,~ \frac{1}{2},~ \frac{dy}{dx}$
%%%%%%%%%%%%%%%%%%%%%%%%%%%%%%%%%%%%%%%%%%%%%%%%%%

\subsection{下付き文字・上付き文字}
$\displaystyle a_n,~e^{i\theta},~\sum_{k=0}^{n}~,~\int_{-\infty}^{\infty}$

%%%%%%%%%%%%%%%%%%%%%%%%%%%%%%%%%%%%%%%%%%%%%%%%%%

\subsection{ベクトルの書き方}
$\vec{v}, \bm{v}$

\subsection{行列とその応用}
\[
A
=
\left(
\begin{array}{cc}
a & b \\
c & d
\end{array}
\right)
=
\left\{
\begin{array}{cl}
B & \text{if} ~~ a > 0 \\
C & \text{else}
\end{array}
\right.
\]

%%%%%%%%%%%%%%%%%%%%%%%%%%%%%%%%%%%%%%%%%%%%%%%%%%
% 2.6 参考文献
%%%%%%%%%%%%%%%%%%%%%%%%%%%%%%%%%%%%%%%%%%%%%%%%%%


\begin{thebibliography}{99}

% urlに日本語が混ざっている場合,URL内の % を \% とする必要があるので注意
\bibitem{github}
片岡凪(2020)
「この文書のソースコード置き場」
\verb|<| \url{https://github.com/KataokaNagi/tex-templates} \verb|>|
2020年12月24日アクセス.

\bibitem{【本の参照名】}
【人名】,【人名】
『【本のタイトル】』
(【出版社】
,2017)

%%%%%%%%%%%%%%%%%%%%%%%%%%%%%%%%%%%%%%%%%%%%%%%%%%
% 2.7 参考文献,2コラム構成,本文の終了地点を設定
%%%%%%%%%%%%%%%%%%%%%%%%%%%%%%%%%%%%%%%%%%%%%%%%%%

\end{thebibliography}
\end{multicols}
\end{document}
