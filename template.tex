% @file           template.tex
% @brief          Template code of LaTeX
% @author         Kataoka Nagi
% @date           2020-12-24 17:19:52
% $Version:       1.0
% $Revision:      1.0
% @note           I will add more comments someday.
% @attention      Use Cloud LaTeX.
% @par            History
%                 New file
% Copyright (c) 2020 Kataoka Nagi
% - This software is released under the MIT License, see LICENSE, see LICENSE.
% - This website content is released under the CC BY 4.0 License, see LICENSE.

\documentclass[a4paper, papersize]{jsarticle}
% [titlepage] でタイトルのみのページになる

\usepackage[dvipdfmx]{graphicx}
\usepackage[]{multicol} % 途中からtwocolumun \begin{multicols}{n}

% for \mathbb{}, \begin{cases}
\usepackage{amsmath, amssymb}
\usepackage{type1cm} 
\usepackage{url}
\usepackage{comment}
\usepackage{mathtools} % for \coloneqq
\usepackage{eqnarray} % 連続するn式
\usepackage{here}

\usepackage{listings,jlisting} % 日本語のコメントアウトをする場合jlistingが必要
% ここからソースコードの表示に関する設定
\lstset{
  basicstyle={\ttfamily},
  identifierstyle={\small},
  commentstyle={\smallitshape},
  keywordstyle={\small\bfseries},
  ndkeywordstyle={\small},
  stringstyle={\small\ttfamily},
  frame={tb},
  breaklines=true,
  columns=[l]{fullflexible},
  numbers=left,
  xrightmargin=0zw,
  xleftmargin=3zw,
  numberstyle={\scriptsize},
  stepnumber=1,
  numbersep=1zw,
  lineskip=-0.5ex
}
% ここまでソースコードの表示に関する設定
% \begin{lstlisting}[caption=hoge,label=fuga] を用いる
% キャプション名「ソースコードn」を「プログラムn」に変えるには,\begin{document}の前に\renewcommand{\lstlistingname}{プログラム}
% https://qiita.com/ta_b0_/items/2619d5927492edbb5b03

% for hyperref
\usepackage[dvipdfmx, bookmarkstype=toc, colorlinks=false, pdfborder={0 0 0}, bookmarks=true, bookmarksnumbered=true]{hyperref}
\usepackage{pxjahyper}

\def\bm#1{\mbox{\boldmath $#1$}} % for bm (vector)

%%%%%%%%%%%%%%%%%%%%%%%%%%%%%%%%%%%%%%%%%%%%%%%%%%%%%%%%%%%%%%%%%%%%%%%%%%%%%%%%

% 目次なし用タイトル
% \title{
% \vspace{-3cm} % タイトルのマージン
% 【授業名】 第【01など】回 レポート課題 \\
% 【サブタイトル】}
% \author{AL18036 片岡 凪}
% \date{提出日 2020年12月18日}


% 目次つき用タイトル
\title{
\vspace{-2cm} % タイトルのマージン
【授業名】 第【01など】回 レポート課題 \\
【サブタイトル】}
\author{AL18036 片岡 凪 \thanks{芝浦工業大学 工学部 情報工学科 3年}}
\date{提出締切 2020年12月23日 \\
提出日  2020年12月18日}

\begin{document}

\maketitle

%%%%%%%%%%%%%%%%%%%%%%%%%%%%%%%%%%%%%%%%%%%%%%%%%%

% 目次
\setcounter{tocdepth}{2}
\tableofcontents
\newpage

%%%%%%%%%%%%%%%%%%%%%%%%%%%%%%%%%%%%%%%%%%%%%%%%%%

\begin{abstract}
【概要】
\end{abstract}

%%%%%%%%%%%%%%%%%%%%%%%%%%%%%%%%%%%%%%%%%%%%%%%%%%%%%%%%%%%%%%%%%%%%%%%%%%%%%%%%

\begin{multicols}{2}
\setcounter{section}{-1}

%%%%%%%%%%%%%%%%%%%%%%%%%%%%%%%%%%%%%%%%%%%%%%%%%%%%%%%%%%%%%%%%%%%%%%%%%%%%%%%%

\section{実行環境}
\subsubsection{java}
\begin{itemize}
\item javac 11.0.7
\item java 11.0.7 2020-04-14 LTS
\item Java(TM) SE Runtime Environment 18.9 (build 11.0.7+8-LTS)
\item Java HotSpot(TM) 64-Bit Server VM 18.9 (build 11.0.7+8-LTS, mixed mode)
\end{itemize}

%%%%%%%%%%%%%%%%%%%%%%%%%%%%%%%%%%%%%%%%%%%%%%%%%%

\subsubsection{ターミナル}
\begin{itemize}
\item cmd.exe
\begin{itemize}
\item Microsoft Windows [Version 10.0.18363.836]上で実行
\end{itemize}
\end{itemize}

%%%%%%%%%%%%%%%%%%%%%%%%%%%%%%%%%%%%%%%%%%%%%%%%%%%%%%%%%%%%%%%%%%%%%%%%%%%%%%%%

\section{設問}

%%%%%%%%%%%%%%%%%%%%%%%%%%%%%%%%%%%%%%%%%%%%%%%%%%

\subsection{設問1}
\begin{quotation}
【本文】\footnote{【氏名】先生の講義,【講義名】の資料「【資料名】」【ページ数】ページ,$<$ \url{【URL】} $>$ (2020年01月17日アクセス) より抜粋}

【本文】\footnote{芝浦工業大学の学生および教職員向けポータルサイトScombの課題提出ページ「【ページ名】」,$<$ \url{【URL】} $>$ (2020年01月17日アクセス) より抜粋} 
\end{quotation}

%%%%%%%%%%%%%%%%%%%%%%%%%%%%%%%%%%%%%%%%%%%%%%%%%%%%%%%%%%%%%%%%%%%%%%%%%%%%%%%%

\section{目的}

%%%%%%%%%%%%%%%%%%%%%%%%%%%%%%%%%%%%%%%%%%%%%%%%%%

\subsection{研究の背景}
【本文】

%%%%%%%%%%%%%%%%%%%%%%%%%%%%%%%%%%%%%%%%%%%%%%%%%%

\subsection{研究の目的}
【本文】

%%%%%%%%%%%%%%%%%%%%%%%%%%%%%%%%%%%%%%%%%%%%%%%%%%%%%%%%%%%%%%%%%%%%%%%%%%%%%%%%

\section{方法}

%%%%%%%%%%%%%%%%%%%%%%%%%%%%%%%%%%%%%%%%%%%%%%%%%%

\subsection{対象}
【本文】

%%%%%%%%%%%%%%%%%%%%%%%%%%%%%%%%%%%%%%%%%%%%%%%%%%

\subsection{装置}
【本文】

%%%%%%%%%%%%%%%%%%%%%%%%%%%%%%%%%%%%%%%%%%%%%%%%%%

\subsection{ソフトウェア}
【本文】

%%%%%%%%%%%%%%%%%%%%%%%%%%%%%%%%%%%%%%%%%%%%%%%%%%

\subsection{ソースプログラム}

%%%%%%%%%%%%%%%%%%%%%%%%%%%%%%

\subsubsection{ソースコード}

\setcounter{lstlisting}{4}
\renewcommand{\lstlistingname}{ソースコード}
\begin{lstlisting}[caption = 【ファイル名】, label=【参照名】]
【ソースコード】
\end{lstlisting}

%%%%%%%%%%%%%%%%%%%%%%%%%%%%%%

\setcounter{lstlisting}{2}
\renewcommand{\lstlistingname}{実行結果}
\subsubsection{実行結果}
\setcounter{lstlisting}{0}
\renewcommand{\lstlistingname}{実行結果}
\begin{lstlisting}[caption = 【ファイル名】 \quad 【ファイル名】, label = 【参照名】]
\end{lstlisting}

%%%%%%%%%%%%%%%%%%%%%%%%%%%%%%%%%%%%%%%%%%%%%%%%%%

\subsection{手続き}
【本文】

%%%%%%%%%%%%%%%%%%%%%%%%%%%%%%%%%%%%%%%%%%%%%%%%%%%%%%%%%%%%%%%%%%%%%%%%%%%%%%%%

\section{仮説}
【本文】

%%%%%%%%%%%%%%%%%%%%%%%%%%%%%%%%%%%%%%%%%%%%%%%%%%%%%%%%%%%%%%%%%%%%%%%%%%%%%%%%

\section{結果}
【本文】

%%%%%%%%%%%%%%%%%%%%%%%%%%%%%%%%%%%%%%%%%%%%%%%%%%%%%%%%%%%%%%%%%%%%%%%%%%%%%%%%

\section{考察}

%%%%%%%%%%%%%%%%%%%%%%%%%%%%%%%%%%%%%%%%%%%%%%%%%%

\subsection{仮説の吟味}
【本文】

%%%%%%%%%%%%%%%%%%%%%%%%%%%%%%%%%%%%%%%%%%%%%%%%%%

\subsection{その他の考察}
【本文】

%%%%%%%%%%%%%%%%%%%%%%%%%%%%%%%%%%%%%%%%%%%%%%%%%%%%%%%%%%%%%%%%%%%%%%%%%%%%%%%%

\section{結論}
【本文】
~\cite{【サイトの参照名】}.

\renewcommand{\figurename}{図}
\setcounter{figure}{2}
\begin{figure}[h]
 \centering
   \includegraphics[width=65mm]{figures/Sample.jpg}
 \caption{【キャプション名】}
 \label{【ラベル名】}
\end{figure}
\noindent

%%%%%%%%%%%%%%%%%%%%%%%%%%%%%%%%%%%%%%%%%%%%%%%%%%%%%%%%%%%%%%%%%%%%%%%%%%%%%%%%

\begin{thebibliography}{99}

% F12 -> alert(document.lastModified); でサイトの更新日がわかる
\bibitem{【サイトの参照名】}
【人名】
(2019)
「【サイトのタイトル】」,
\verb|<| \url{【URL】} \verb|>|
2020年01月17日アクセス.

\bibitem{【本の参照名】}
【人名】,【人名】
『【サイトのタイトル】』
(【出版社】
,2017)

\end{thebibliography}

%%%%%%%%%%%%%%%%%%%%%%%%%%%%%%%%%%%%%%%%%%%%%%%%%%

\begin{comment}
メモ
\end{comment}

\end{multicols}

\end{document}
