% @file           tex_practice.tex
% @brief          Practice code of LaTeX
% @author         Kataoka Nagi
% @date           2020-12-24 17:41:39
% $Version:       1.0
% $Revision:      1.0
% @note           I will add more comments someday.
% @attention      Use Cloud LaTeX.
% @par            History
%                 New file
% Copyright (c) 2020 Kataoka Nagi
% - This software is released under the MIT License, see LICENSE, see LICENSE.
% - This website content is released under the CC BY 4.0 License, see LICENSE.

% [uplatex]で文字化け防止
% papersize: pdfを用紙サイズと合わせる 
% article report book  reportはレポートというより長い報告書
% [titlepage]: タイトルで1ページ取る
\documentclass[a4paper,12pt,papersize,twocolumn,titlepage]{jsarticle}

% プリアンブル:\documentclass~\document

% Cloud LaTeX で弄るとエラーになる?
% {otf}で文字化け防止
% {newpxtext} でフォントがPalainoとHelveticaに
% [utf8]
\usepackage[dvipdfmx]{graphicx}

\title{レポートタイトル \\ 長いタイトルは\\で改行}

%ページ番号無効
\pagestyle{empty}

\author{学生番号AL18036 片岡凪 \and \andで2人目 \thanks{\thanks芝浦大 で著者脚注へ(著者の間に\\でも可)}}

\date{\today}

\begin{document}

% \title の後に記述
\maketitle 

% 目次
\setcounter{tocdepth}{2}
\tableofcontents

\begin{abstract}
\maketatile の後に\begin{abstract}\end{abstract}で概要が載る
\end{abstract}


\section{Cloud LaTeXへようこそ}

Cloud LaTeXは,\LaTeX を使った文書の作成・管理をクラウド上で行えるWebサービスです.
\LaTeX を使うと,複雑な数式
\begin{equation}
  \frac{\pi}{2} =
  \left( \int_{0}^{\infty} \frac{\sin x}{\sqrt{x}} dx \right)^2 =
  \sum_{k=0}^{\infty} \frac{(2k)!}{2^{2k}(k!)^2} \frac{1}{2k+1} =
  \prod_{k=1}^{\infty} \frac{4k^2}{4k^2 - 1}
\end{equation}
を含んだ読みやすくきれいな文書作成ができます.

\newpage

本サービスは,\LaTeX 文書をリアルタイムに保存・コンパイルし,ユーザーアカウント別に管理します.
そのため,本サービスにログインするだけで,どこからでも作業を再開でき,ファイルを持ち歩く必要はありません.
また,様々な \LaTeX テンプレートが用意されているので,手軽に文書を作り始めることができます.

\begin{figure}
 \centering
   \includegraphics[width=30mm]{figures/Sample.png}
 \caption{ここにキャプションを挿入します}
 \label{fig:model}
\end{figure}

\noindent
Cloud LaTeXでは,作成されるPDFそのままのレイアウトで表示するPDFビューモードがあり,コンパイル画面を確認しながら文書を作成することができます(図\ref{fig:model})
日本語では,pLaTeX / LuaLaTeX / upLaTeX でのコンパイルが可能です.
また,日本語や英語文書作成だけでなく,中国語・ハングルに対応した XeLaTeX のコンパイルも可能です.
ぜひ使ってみてください.

\part{部見出し}
% \chapter{章見出し} % jsarticleなどにはない
\section*{section*: セクションから数字を削除}
\subsection{サブセクション}
\subsubsection{サブサブセクション}
\paragraph{段落見出し}
\subparagraph{サブ段落見出し}
\begin{quotation}
引用
\end{quotation}

\begin{verbatim}
\begin{verbatim}
\end{verbatim}
で記号を文字通りに出力:-)
\end{verbatim}

・数文字程度なら\verb||で可能 \verb|^^;|

|でなくてもよい \verb">w<" \verb->w<-

\verb*||でスペースが\verb*" "になる


・英字での改行は空白になる

like

this

↓

like
this


・%で改行無視

like%
this


・\\verb*" "で\ \ \ \ 連続空白


・~で改行なし空白(~~~~~~~~)


・\$ などは\\$


・アクセント文字\^{a}などは\verb"\^{a}"


・区切り---は\verb"---"


・\verb"\textgt{}"\textgt{でゴシック体になる}
・\verb"\textit{}"\textit{でイタリック体になる(はず)}

・\verb*"{\Large }" で{\Large 大きな}文字など


・\verb"\begin{}\end{} を環境という"


・\verb"\begin{flushleft}左寄せ\end{flushleft}"

\begin{flushleft}
左寄せ
\end{flushleft}


・\verb"\begin{center}中央寄せ\end{center}"

\begin{center}
中央寄せ
\end{center}


\begin{verbatim}
\begin{itemize}
\item item
\end{itemize}
\end{verbatim}
で

\begin{itemize}
\item 箇条書き
%\begin{itemize}
%\item item
%\end{itemize}
\end{itemize}


\begin{verbatim}
\begin{enumerate}
\item item
\end{enumerate}
\end{verbatim}
で

\begin{enumerate}
\item 番号付き箇条書き
\end{enumerate}


・\verb"/footnote{脚注}"
\footnote{脚注}


・\verb"\marginpar{欄外}"
\marginpar{欄外}


・\verb"\reversemarginpar{反対の欄外}"{}
%\reversemarginpar{反対の欄外}


・\verb"\underline{}" で\underline{アンダーライン} 


・\verb"\hrulefill"

\hrulefill


・\verb"\dotfill"

\dotfill


・\verb"\fbox{fbox}"
で
\fbox{fbox}


・\verb"\framebox[2cm]{fbox}"
で
\framebox[2cm]{fbox}


・
\begin{verbatim}
\begin{equation}
E=mc^2 \label{emc}
\end{equation}
\end{verbatim}
で
\begin{equation}
E=mc^2 \label{emc}
\end{equation}

・\verb"\sum_{k=0}^1" で
$\sum_{k=0}^1$


・\verb"式\ref{emc}によると"
=式\ref{emc}によると

・文章中で使うなら\verb"$\displaystyle \sum_{k=0}^1$" 
ほげ $\displaystyle \sum_{k=0}^1$ ふが

・\verb"$\bigl((a+b)+c\bigr)+d$"で

$\bigl((a+b)+c\bigr)+d$

・\verb""


・\verb"\frac{b}{a}" で
$\frac{b}{a}$


\begin{verbatim}
\usepackage{amsmath, amssymb}
\usepackage{type1cm}
\mathbb{N, Z, Q, R, C}
\end{verbatim}

\begin{verbatim}

\end{verbatim}


・奥村~\cite[181--204ページ]{latex2e,statistics}によると、\verb"~\cite[181--204ページ]{latex2e,statistics}"で参考文献を表示できる。
% \usepackage{cite} で連続参照で1-3となる
\begin{thebibliography}{99}
\bibitem{latex2e}
奥村晴彦, 黒木裕介『LATEX2ε美文書作成入門』
(技術評論社, 2017)
\bibitem{statistics}
石綿元(2019)「統計学(確率と統計2)講義(確率と統計第3・確率と統計第4)<2019年度>」,
\begin{verbatim}
<https://lecture.neocities.org/> 2020年1月17日アクセス.
\end{verbatim}
\end{thebibliography}{99}


\end{document}
